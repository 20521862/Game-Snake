\documentclass{article}
\usepackage[utf8]{vietnam}
\usepackage[english]{babel}
\usepackage{diagbox}
\usepackage{array}
\usepackage[left=3.00cm, right=1.5cm, top=2.00cm, bottom=2.00cm]{geometry}
\usepackage[parfill]{parskip}
\usepackage{indentfirst}
\usepackage{hyperref}

\title{SS004.9}
\author{Hợp đồng nhóm}
\date{5/2022}

\renewcommand\thesection{\Roman{section}}
\renewcommand\thesubsection{\arabic{subsection}}



\begin{document}

\maketitle


\section{Tên nhóm: ALT}

\section{Tên thành viên:}

\begin{center}
\begin{tabular} {|c|c|c|} \hline
STT & Tên & MSSV \\\hline
1 & Trần Tấn Tài & 20521862 \\ \hline
2 &  Lê Tuấn Lương & 20521588 \\ \hline
3 &  Nguyễn Thiên Ân & 20521054 \\ \hline

\end{tabular}
\end{center}

\section{Mục tiêu của nhóm:}


\begin{enumerate}
\item Ra quyết định dựa trên đa số.
\item Chấp hành tốt nội quy, kỷ luật.
\item Rèn luyện và nâng cao kỹ năng làm việc nhóm của từng thành viên trong nhóm.
\item Tạo mối quan hệ đoàn kết, gắn bó, cống hiến giữa các thành viên sau khóa học, nhằm phát huy năng lực và khả năng sáng tạo của mỗi thành viên, tăng cường tương tác lẫn nhau, tự học.
\item Hoàn thành các yêu cầu về thời gian làm việc, làm việc nhóm, nâng cao kiến thức và kỹ năng sử dụng phần mềm (như Latex, Trello, Github v.v.), nâng cao khả năng tìm kiếm và tổng hợp thông tin...
\item Kiến thức tiếp thu được áp dụng cho đồ án, khóa luận và hoàn thành khóa học với số điểm cao (tối thiểu 70\% tổng điểm).
\end{enumerate}



\section{Vai trò các thành viên trong nhóm:}

\begin{center}
\begin{tabular} {|p{3.9cm}|>{\raggedright\arraybackslash}p{4.1cm}|>{\raggedright\arraybackslash}p{2.1cm}|>{\raggedright\arraybackslash}p{2cm}|>{\raggedright\arraybackslash}p{2.9cm}|} \hline
\diagbox{Thành viên}{Vai trò} & Lãnh đạo nhóm và giữ cho công việc đi đúng hướng & Thiết kế giao diện game & Tester và viết báo cáo&Viết code các chức năng của game \\\hline
Trần Tấn Tài & X & & X & \\ \hline
Lê Tuấn Lương & & X & & X \\ \hline
Nguyễn Thiên Ân & & X & & X \\ \hline

\end{tabular}
\end{center}

\section{Phiếu đánh giá:}

\begin{center}
\begin{tabular} {|p{3cm}|>{\raggedright\arraybackslash}p{2.4cm}|>{\raggedright\arraybackslash}p{2.4cm}|>{\raggedright\arraybackslash}p{2.4cm}|>{\raggedright\arraybackslash}p{2.4cm}|} \hline
\diagbox{Tiêu chí}{Xếp loại} & Xuất sắc & Giỏi & Trung bình & Kém \\\hline

Khả năng tìm kiếm thông tin & Luôn chủ động tìm kiếm thông tin & Thường tìm kiếm thông tin & Chỉ hoàn thành khi được nhắc & Không bao giờ \\ \hline

Tinh thần hợp tác & Luôn giữ liên lạc với nhóm & Thường giữ liên lạc với nhóm & Lúc ẩn lúc hiện & Không thể liên lạc \\ \hline 

Tinh thần trách nhiệm & Luôn gửi bài tập trước thời hạn & Thường nộp bài tập đúng giờ & Thường nộp bài tập muộn vài phút với thời gian đã định &Luôn gửi bài tập quá hạn hoặc không gửi bài tập\\ \hline

Giải quyết các vấn đề phát sinh & Luôn chủ động giải quyết vấn đề một cách cẩn thận & Thường giải quyết các vấn đề một cách cẩn thận &Luôn giải quyết vấn đề theo cách bình thường & Không bao giờ giải quyết vấn đề \\ \hline

Quản lý thời gian & Luôn có mặt sớm cho các cuộc họp nhóm & Thường có mặt sớm cho các cuộc họp nhóm & Đôi khi xuất hiện muộn 5-10 phút trong các cuộc họp nhóm & Luôn đi họp nhóm muộn \\ \hline

\end{tabular}
\end{center}
\begin{itemize}
\item Phương thức đánh giá: Trưởng nhóm sẽ đánh giá các thành viên trong nhóm
\end{itemize}

\section{Nơi làm việc:}
\begin{itemize}
\item Link github:
\url{https://github.com/20521862/Game-Snake}
\item{Ngoài ra có thể nhắn tin trực tiếp trên group messenger.}
\end{itemize}

\section{Hợp đồng:}
\begin{center}
\begin{tabular}{p{4cm}p{4cm}p{4cm}}
Trần Tấn Tài & Lê Tuấn Lương & Nguyễn Thiên Ân  \\
     &  & \\
             \\
.........................& .........................& ......................... \\     
\end{tabular}   
\end{center}


\newpage
\section{Giới thiệu và hướng dẫn cách chơi game:}
\Large
\begin{enumerate}
    \item Giới thiệu:\\
    Snake là một thể loại trò chơi điện tử trong đó người chơi điều khiển một đường đang phát triển trở thành chướng ngại vật chính đối với chính nó. Khái niệm này bắt nguồn từ trò chơi điện tử hai người chơi Blockade năm 1976 của Gremlin Industries, và sự dễ dàng thực hiện đã dẫn đến hàng trăm phiên bản (một số phiên bản có từ rắn hoặc sâu trong tiêu đề) cho nhiều nền tảng. Trò chơi arcade Tron của năm 1982, dựa trên bộ phim, bao gồm trò chơi rắn cho phân đoạn Light Cycles một người chơi. Sau khi một biến thể được cài đặt sẵn trên điện thoại di động Nokia vào năm 1998, sự quan tâm đến trò chơi rắn lại nổi lên khi nó tìm được lượng người xem lớn hơn\\
    Thiết kế Snake có từ thời trò chơi arcade Blockade, được phát triển và xuất bản bởi Gremlin vào năm 1976. Nó được nhân bản thành Bigfoot Bonkers cùng năm. Năm 1977, Atari, Inc. đã phát hành hai tựa game lấy cảm hứng từ Blockade: trò chơi arcade Dominos và trò chơi Atari VCS Surround. Surround là một trong chín tựa game ra mắt Atari VCS tại Hoa Kỳ và được bán bởi Sears dưới tên Chase. Cùng năm đó, một trò chơi tương tự đã được ra mắt cho Bally Astrocadenhư Checkmate.\\
    Phiên bản máy tính gia đình đầu tiên được biết đến , có tên Worm , được lập trình vào năm 1978 bởi Peter Trefonas cho TRS-80, và được xuất bản bởi tạp chí CLOAD trong cùng năm. Điều này được tiếp nối ngay sau đó với các phiên bản của cùng một tác giả cho Commodore PET và Apple II. Một bản sao của trò chơi arcade Hustle, bản thân nó là một bản sao của Blockade, được viết bởi Peter Trefonas vào năm 1979 và được xuất bản bởi CLOAD. Một phiên bản được ủy quyền của Hustle đã được Milton Bradley xuất bản choTI-99 / 4A vào năm 1980. Snake Byte chơi đơn được xuất bản vào năm 1982 cho các máy tính 8-bit Atari, Apple II và VIC-20; một con rắn ăn táo để hoàn thành một cấp độ, phát triển lâu hơn trong quá trình này. Trong Snake for the BBC Micro (1982) của Dave Bresnen, con rắn được điều khiển bằng cách sử dụng các phím mũi tên trái và phải so với hướng nó đang đi tới. Con rắn tăng tốc độ khi nó dài hơn và chỉ có một mạng sống.\\
    Nibbler (1982) là một trò chơi arcade dành cho một người chơi, trong đó con rắn nằm gọn trong một mê cung và lối chơi nhanh hơn hầu hết các thiết kế rắn. Một phiên bản chơi đơn khác là một phần của trò chơi arcade Tron năm 1982 nó làm sống lại khái niệm con rắn và nhiều trò chơi tiếp theo đã mượn chủ đề chu kỳ ánh sáng.\\
    Bắt đầu từ năm 1991, Nibbles đã được đưa vào MS-DOS trong một khoảng thời gian như một chương trình mẫu QBasic. Năm 1992, Rattler Race được phát hành như một phần của Microsoft Entertainment Pack thứ hai . Nó bổ sung thêm rắn kẻ thù vào lối chơi ăn táo quen thuộc.
    \item Cách chơi:\\
    Khi vào ta thấy menu có ba lựa chọn:\\
     a. Easy (Chế độ không có tường chắn)\\
     b. Normal (Chế độ tường chắn bao quanh)\\
     c. Hard(Chế độ tường chắn thu nhỏ phạm vi theo thời gian)\\
     Mỗi chế độ đều có 3 cơ chế điều chỉnh tốc độ khác nhau (chậm, bình thường, nhanh).\\
     Người chơi có thể lựa chọn chế độ bằng cách sử dụng các phím di chuyển (←↑↓→) để lựa chọn chế độ thích hợp sau đó nhấn enter để vào trải nghiệm chế độ đó. Để tạm dừng trò chơi ta nhấn phím cách (space) trên bàn phím.\\
     Khi vào chơi game, người chơi đóng vai thành con rắn săn mồi. Khi ăn được mồi chiều dài rắn sẽ dài ra, và điểm người chơi sẽ được cộng thêm 10đ. Khi chiều rắn càng dài thì tương đương về độ khó càng cao, do khi đầu rắn chạm vào thân rắn thì người chơi sẽ bị xử thua cuộc. Ngoài ra nếu rắn chạm vào tường cũng sẽ dẫn đến việc thua cuộc.\\
     Người chơi sẽ sử dụng các phím A(qua trái), W(lên trên), S(xuống dưới), D(qua phải) để điều khiển chú rắn của mình, sao cho tránh được việc đầu rắn chạm vào thân rắn hoặc tường mà vẫn ăn được mồi để tích được thật nhiều điểm và chiến thắng trò chơi.\\
     Ngoài ra ở chế độ đặc biệt thì trò chơi sẽ được thiết kế chế độ đặc biệt, tường sẽ thu nhỏ dần sau mỗi 20s để thu hút người chơi.\\
\end{enumerate}

\newpage
\section{Tài liệu kỹ thuật của trò chơi:}

Khi mở game snake và chọn play game ta sẽ xuất hiện menu để lựa chọn chế độ chơi. Chế độ 1 là chế độ cổ điển, được xem là dễ nhất do không có tường chắn làm vật cản nên giảm khả năng thua, thích hợp với người mới tập chơi. Chế độ thứ 2 là chế độ có tường chắn bao quanh nên đòi hỏi người chơi phải có tính tỉ mỉ để có thể ăn những thức ăn ở sát bên tường. Chế độ 3 là chế độ tường chắn sẽ thu nhỏ dần sau một thời gian cố định, đòi hỏi người chơi vừa phải có tính tỉ mỉ vừa phải tính toán thời gian chính xác để không phải dừng cuộc chơi sớm. Mỗi chế độ điều có 3 mức chỉnh tốc độ khác nhau (nhanh, bình thường, chậm) để tăng tính đa dạng cho game. Ngoài ra game còn có chức năng tạm dừng và tiếp tục khi đang trong chế độ chơi.\\
Chương trình gồm 3 phần chính: là Snake(tác động lên rắn), Zone(tác động đến bản đồ trò chơi), Food(tác động đến thức ăn cho rắn). \\
Struct NODE để cấu trúc nên tọa độ nhằm xác định vị trí của rắn. \\
Trong Class Snake có:
\begin{itemize}
\item Hàm khởi tạo Snake() dùng để khởi tạo chiều dài và vị trí đầu tiên mà rắn xuất hiện.
\item Hàm Snake.Draw dùng để thiết kế hình dạng rắn (Design). 
\item Hàm Snake.Move dùng để điều khiển hướng di chuyển của rắn. 
\end{itemize}
Trong Class Zone có: 
\begin{itemize}
\item Hàm khởi tạo Zone() dùng để khởi tạo chiều dài và chiều rộng của bản đồ. 
\item Hàm Zone.Draw dùng để thiết kế màu sắc cho bản đồ (Design). 
\end{itemize}
Trong Class Food: 
\begin{itemize}
\item Hàm khởi tạo Food() dùng để khởi tạo thức ăn. 
\item Hàm Food.Random dùng để random vị trí xuất hiện của thức ăn.
\item Hàm Food.Draw dùng để thiết kế hình dạng thức ăn. 
\end{itemize}
Ngoài ra trong main.cpp còn có một số hàm cần thiết như: 
\begin{itemize}
\item Hàm Update() dùng để cập nhật liên tục trạng thái của rắn, nếu rắn chạm tường hoặc thì sẽ Game over, còn ngược lại thì tiếp tục chạy chương trình. 
\item Hàm DrawScore() dùng để tính điểm cho người chơi.
\item Hàm DrawTime() dùng để tính giờ. 
\item Hàm selectspeed() dùng để set tốc độ cho rắn. Và gọi đến hàm playgame().
\item Hàm gameover() hàm này được gọi khi hàm Update() trả về giá trị False (rắn chạm tường hoặc chạm thân).
\item Hàm drawPause() dùng để thực hiện chức năng tạm dừng trong khi chơi trò chơi.
\item Hàm playgame() dùng để khởi tạo trò chơi nó sẽ gọi đến các hàm khỏi tạo khác như: Snake.Draw(), Food.Draw(), drawScore(), drawTime(), drawPause()…
\item Hàm selectmode() dùng để lựa chọn chế độ chơi game.
\item Hàm Start() dùng để mở menu đầu tiên. Khi ấn chọn play game thì nó sẽ gọi đến hàm selectmode(), nếu chọn exit thì nó sẽ gọi đến hàm Thank().
\item Hàm Thank() dùng để hiển thị tên thành viên của nhóm, và lời cảm ơn đến mọi người đã theo dõi phần demo của nhóm.
\end{itemize}



\newpage
\section{Mô tả quá trình làm việc nhóm:}
\Large

Sau khi nhận được đề tài, nhóm trưởng phân chia công việc cho từng thành viên nhóm theo cách thức: nhóm trưởng sau khi đọc đề tài thì chia công việc thành 3 phần chính. Một là Tester và soạn thảo báo cáo. Hai là các chức năng chính của game. Ba là thiết kế lại giao diện của game sau ghi đã nhận được code chức năng game. Sau khi đã phân chia rõ công việc, nhóm trưởng đề nghị các thành viên chọn công việc trước để đảm bảo tính công bằng, công việc còn lại sẽ do người phân chia chịu trách nhiệm. Ngoài ra sẽ phân công thêm người hỗ trợ, đề phòng trường hợp không thể hoàn thành như dự kiến bởi một số lí do khách quan.\\
Ở tuần thứ nhất của đồ án, các thành viên cùng tìm kiếm tài liệu và học sử dụng GitHub để có thể cùng thực hiện đồ án. Mỗi thành viên tạo cho mình 1 tài khoản trên GitHub theo yêu cầu của giảng viên hướng dẫn.Trưởng nhóm tạo dự án trên GitHub với tên Game-Snake. Sau đó nhóm trưởng thêm giảng viên hướng dẫn vào team trên GitHub. Ngoài trao đổi trên GitHub, trường nhóm còn tạo nhóm messenger trên facebook, mục đích để các thành viên có thể trao đổi thông tin một cách nhanh chóng, do môi trường nhắn tin thân thiện và dễ sử dụng nên các thành viên thường xuyên trao đổi thông tin để hổ trợ lẫn nhau. Do tuần thứ nhất giảng viên chỉ yêu cầu nộp hợp đồng nhóm, nên tester đã soạn thảo hợp động sau đó gửi cho các thành viên khác kiểm tra, sau khi hợp đồng đã được thống nhất với nhau thì tester sẽ chịu trách nhiệm nộp hợp đồng nhóm cho giảng viên. Đồng thời trưởng nhóm cũng sẽ đăng các công việc cần làm trên GitHub.\\
Ở tuần thứ hai, trưởng nhóm code mã nguồn cơ bản như giảng viên đã hướng dẫn, sau đó đưa lên GitHub để các thành  viên cùng xem và thảo luận phương hướng tiến hành làm đồ án, đề ra các chức năng cần thiết cho đồ án lần này. Các thành viên trong nhóm thay nhau đưa ra ý kiến sau đó biểu quyết để chọn ra một vài chức năng cơ bản để người viết code viết trước. Sau đó người thiết kế giao diện dựa trên đoạn code đó tiến hành thiết giao diện lần thứ nhất, sau đó chạy thử cho cả nhóm cùng xem trong cuộc họp lần thứ nhất của nhóm. Sau đó tester sẽ dựa vào phần chạy thử của người thiết kế giao diện và source code hiện có, để đề ra các yêu cầu tiếp theo để đồ án trở nên hoàn thiện hơn. Sau khi test lần một đồ án đã đạt yêu cầu về một số chức năng cơ bản. Game đã có thể chơi ở mức cơ bản như ăn mồi sẽ khiến chiều dài rắn dài ra và tính điểm, ngoài ra cũng đã có cơ chế game over khi đầu rắn chạm vào thân rắn hay đầu rắn chạm vào vật cản. Phần test game lần 1 đến đây xem như kết thúc. Trưởng nhóm soạn lại hợp đồng nhóm dựa trên những góp ý, nhận xét của giảng viên đối với hợp đồng nộp lần thứ nhất, đồng thời đính kèm link GitHub của dự án cũng và các link của nơi nhóm làm việc, sau đó gửi các thành viên kiểm tra lại lần cuối, sau khi đã thống nhất thì tiến hành nộp lại hợp đồng nhóm cho giảng viên.\\
Ở đầu tuần thứ ba, trưởng nhóm mở cuộc họp lần thứ hai để đề ra những mục tiêu cần thực hiện trong tuần này và đốc thúc các thành viên các nhanh chóng chuẩn bị tài liệu cần thiết dành cho nhiệm vụ của mình, để đảm bảo hoàn thành công việc đúng thời gian đã định. Tester có những commit cụ thể để người lập trình  có thể chuyển những ý tưởng của tester thành code thực tế. Sau khi người lập trình nhận được commit đã nhanh chóng tìm kiếm tài liệu để có thể bổ sung code cần thiết nhằm hiện thực ý tưởng của tester. Sau khi đã nhận được đoạn code ý tưởng đầu tiên của mình mong đợi, tester đã yêu cầu người lập trình viết thêm 2 chế độ chơi game nữa, để tăng tính hấp dẫn và cũng còn nhiều thời gian chuẩn bị cho việc báo cáo đồ án. Cụ thể là chế độ không có vật cản và chế độ tính giờ để thu nhỏ bản đồ lại nhằm cho game có một độ khó mới, hấp dẫn người chơi.Trong khoảng thời gian này tester và người lập trình đã tiến hành sửa đổi code rất nhiều lần, cụ thể là qua các commit của tester trên GitHub để có thể tối ưu đồ án game snake này. Sau khỉ đã hoàn thiện các chức năng, trưởng nhóm đã mở cuộc hợp lần thứ ba, để tiến hành chảy thử code cho các thành viên cùng kiểm tra, sau khi lấy ý kiến thì các thành viên đã quyết định chính thức sử dụng đoạn code này và chuyển sang phần thiết kế giao diện game. Ở phần thiết kế giao diện game này, người thiết kế đã gặp một vài khó khăn do lần đầu nhận nhiệm vụ này, cụ thể là ở phần điều chỉnh các cài đặt để có thể thiết kế giao diện. Nhưng sau đó nguời thiết kế đã nhanh chóng làm quen dần và bắt đầu thực hiện công việc của mình. Sau 2 ngày thì người thiết kế giao diện đã cho bản chạy thử lần thứ nhất, và trình chiếu cho cả nhóm cùng kiểm tra, sau đó tester đã yêu cầu người thiết kế điều chỉnh một vài chi tiết để tăng tính thẩm mỹ. Sau khi người thiết kế giao diện kiểm tra lại code của người lập trình thì thấy để chung 1 file code thì khó thiết kế, nên người thiết kế đã yêu cầu người lập trình tách code ra thành từng class riêng để thuận tiện cho việc thiết kế, sau khi nhận được yêu cầu thì người lập trình đã tiến hành sửa code ngay, do thành thạo đoạn của mình nên người lập trình đã hoàn thành việc tách ra chỉ trong vài giờ đồng hồ. Sau khi đã nhận được đoạn code mong đợi, người thiết kế giao diện đã bắt đầu thiết kế, do gặp phải một số khó khăn nên người lập trình đã hổ trợ người thiết kết một vài việc cần thiết để đảm bảo tiến độ đồ án, do gặp một số lí do bản quyền nên nhóm  đã quyết định lấy hình ảnh đơn giản làm background. Sau 2 ngày các thành viên làm việc hết sức cùng nhau, test và fix bug liên tục thì cũng đã có đoạn code hoàn chỉnh. Sau đó nhóm trưởng đã mở một cuộc họp để demo chương trình và đề ra phương hướng vấn đáp của nhóm, ngoài ra còn thảo luận để sửa trực tiếp tại lúc demo một vài chi tiết nhỏ trong code để code trở nên hoàn chỉnh nhất, sau đó kết thúc việc viết code và design tại đây. Và tester tiến hành viết báo cáo để nộp giảng viên.\\



\section{Kỹ năng đã áp dụng:}

\begin{itemize}
    \item Kỹ năng làm việc nhóm.
    \item Kỹ năng giao tiếp.
    \item Kỹ năng thuyết trình.
    \item Kỹ năng quản lí thời gian.
    \item Kỹ năng tìm kiếm tài liệu
    \item Kỹ năng đọc hiểu tài liệu tiếng anh cơ bản.
    \item Kỹ năng sử dụng GitHub để quản lý dự án.
    \item Kỹ năng sử dụng visual studio code kết hợp với GitHub.
    \item Kỹ năng vấn đáp, phản biện.
    \item Kỹ năng sáng tạo.
    \item Kỹ năng tư duy logic.
    \item Kỹ năng giải quyết vấn đề.
    \item Tư duy hệ thống
    \item Kỹ năng viết báo cáo.
    \item Kỹ năng chịu áp lực công việc.
    \item Kỹ năng tự học.
    \item Kỹ năng lập trình game bằng C++.
    \item Kỹ năng thiết kế giao diện game.
    \item Kỹ năng xử lí tình huống bất ngờ.
    \item Kỹ năng thích nghi nhanh với công việc.
    
\end{itemize}

\newpage
\section{Đánh giá:}


\begin{center}
\begin{tabular} {|p{4.5cm}|>{\raggedright\arraybackslash}c|>{\raggedright\arraybackslash}c|>{\raggedright\arraybackslash}c|} \hline
\diagbox{Tiêu chí}{Sinh viên} & Trần Tấn Tài & Lê Tuấn Lương & Nguyễn Thiên Ân \\\hline

Khả năng tìm kiếm thông tin & Xuất sắc & Xuất sắc & Xuất sắc \\ \hline

Tinh thần hợp tác & Xuất sắc & Xuất sắc & Xuất sắc \\ \hline 

Tinh thần trách nhiệm & Xuất sắc & Xuất sắc & Xuất sắc \\ \hline

Giải quyết các vấn đề phát sinh & Xuất sắc & Xuất sắc &Xuất sắc \\ \hline

Quản lý thời gian & Xuất sắc & Xuất sắc &Xuất sắc \\ \hline

\end{tabular}
\end{center}

\end{document}
